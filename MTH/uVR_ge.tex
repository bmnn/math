% [p. 499]
Eine nichtleere Teilmenge $\mathcal{U}$ eines $\mathbb{K}$ Vektorraums $\mathcal{V}$ heisst Untervektorraum von $V$ wenn gilt:
\begin{enumerate}
	\item $u,w \in \mathcal{U} \implies u + w \in \mathcal{U}$ 
	\item $\lambda \in \mathbb{K}, u \in \mathcal{U} \implies \lambda u \in \mathcal{U}$ 
\end{enumerate}


