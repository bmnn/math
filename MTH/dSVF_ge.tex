Eine Funktion $\operatorname{f} : \mathbb{R}^n \to \mathbb{R}$ ist dann im Punkt $\tilde{x}$ differenzierbar wenn
$$\exists a : \operatorname{f}(x) = \operatorname{f}(\tilde{x}) + a(x - \tilde{x}) + \operatorname{R}(x), \lim\limits_{x \to \tilde{x}} \frac{\operatorname{R}(x)}{|| x - \tilde{x} || } = 0$$
Die Funktion $\operatorname{f}$ heisst differenzierbar in $\mathcal{B} \subset \mathbb{R}^n$, wenn sie in jedem Punkt $\tilde{x} \in \mathcal{B}$ differenzierbar ist.

