% [p. 510]
Fuer einen $\mathbb{K}$ Vektorraum $\mathcal{V} : \operatorname{dim}(\mathcal{V}) \geq 2$ mit Untervektorraum $\mathcal{U}$ und jedes $v \in \mathcal{V}$ wird
$$\mathcal{T} = v + \mathcal{U} = \left\{ v + x | x \in \mathcal{U} \right\}$$ 
ein affiner Teilraum von $\mathcal{V}$ genannt. Es heissen $\mathcal{\mathcal{U}}$ seine Richtung und $\operatorname{dim}(\mathcal{T}) = \operatorname{dim}(\mathcal{U})$  seine Dimension. Affine Teilraeumen der Dimension $1$ bzw. $2$ heissen Geraden bzw. Ebenen.  
