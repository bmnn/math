% [p. 855]
Fuer ein Gebiet $\mathcal{D} \in \mathbb{R}^{2}$, eine Lebesgue integrierbare Funktion $\operatorname{f}(.) \in \mathcal{L}(\mathcal{D})$ und $\mathcal{B}$ als Beschreibung von $\mathcal{D}$ in Polarkoordinaten ist
$$\int\limits_{\mathcal{D}} \operatorname{f}(x_{1}, x_{2}) \operatorname{d} x_{1}  \operatorname{d} x_{2} = \int\limits_{\mathcal{B}} r \operatorname{f}(r \cos(\varphi), r \sin(\varphi)) \operatorname{d} r \operatorname{d} \varphi$$
