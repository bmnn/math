% [p. 27]
Es seien $\mathcal{V}, \mathcal{W}$ zwei Vektorraeume.  Jede eindeutige Abbildung $f : \mathcal{V} \to \mathcal{W}$ heisst lineare Abbildung (auch Transformation, Operator) wenn gilt
\begin{enumerate}
	\item $\forall u,v \in \mathcal{V} : \operatorname{f}(u + v) = \operatorname{f}(v) + \operatorname{f}(v)$ 
	\item $\forall u \in \mathcal{V}, \lambda \in \mathbb{R} : \operatorname{f}(\lambda u) = \lambda \operatorname{f}(u)$ 
\end{enumerate}

