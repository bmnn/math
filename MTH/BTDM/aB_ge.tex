% [p. 49]
Gegeben sind zwei nicht leere Mengen $\mathcal{X}$ und $\mathcal{Y}$. Unter einer Abbildung die durch 
$$\operatorname{f} : \mathcal{X} \to \mathcal{Y}$$ 
beschrieben wird, versteht man eine Vorschrift, die jedem Element $x \in \mathcal{X}$ eindeutig ein Element $y \in \mathcal{Y}$ zuordnet.  Das Element $y$ heisst Bild von $x$ und man schreibt auch $y = \operatorname{f}(x)$.  Die Menge $\mathcal{Y}$ heisst Bildbereich ('codomain') oder Wertebereich('target set') von $\operatorname{f}$, die Menge $\mathcal{X}$ heisst Originalbereich('domain'), Urbildbereich oder Definitionsbereich von $\operatorname{f}$.  
