% [p. 812]
% Kettenregel fuer mehrdimensionale Funktionen
Fuer die beiden ueber offenen Mengen $\mathcal{X}, \mathcal{Y}$ definierten Funktionen
$$\operatorname{f} : \mathcal{X} \subset \mathbb{R}^n \to \mathbb{R}^q$$ 
$$\operatorname{g} : \mathcal{Y} \subset \mathbb{R}^q \to \mathbb{R}^m, \operatorname{f}(\mathcal{X}) \subset \mathcal{Y}$$
wobei $\operatorname{f}$ im Punkt $p$ und $\operatorname{g}$ im Punkt $q = \operatorname{f}(p)$ differenzierbar ist, ist die Funktion
$$\operatorname{h} : \mathcal{X} \to \mathbb{R}^m$$ 
mit $\operatorname{h} = \operatorname{g} \circ \operatorname{f}$ differenzierbar und es gilt
$$\frac{\partial (\operatorname{h}_1, \dots \operatorname{h}_m)}{\partial (x_1, \dots, x_n)} \bigg|_p = \frac{\partial (\operatorname{g}_1, \dots \operatorname{g}_m)}{\partial (y_1, \dots, y_q)} \bigg|_q  \frac{\partial (\operatorname{f}_1, \dots \operatorname{f}_q)}{\partial (x_1, \dots, x_n)} \bigg|_p$$ 
