Ist $\mathcal{V}$ ein $n$ dimensionaler Vektorraum ueber $\mathbb{K}$, so nennt man eine Funktion $|| . || : \mathcal{V} \to \mathbb{R}_{\geq 0}$ eine Norm, falls die folgenden Eigenschaften fuer $u,v \in \mathcal{V}$, $\lambda \in \mathbb{K}$ erfuellt sind.
\begin{enumerate}
  \item $|| v || = 0 \iff v = 0^n$
  \item $|| \lambda v || = |\lambda| || v ||$
  \item $|| u + v || \leq || u || + || v ||$ % Dreiecksungleichung ('triangle inequality')
\end{enumerate}
