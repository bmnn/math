In der Darstellung der Uebertragungsfunktion $\operatorname{G}(s)$ eines Systems als gebrochenrationale Funktion von $\operatorname{G}(s) = \frac{\operatorname{Z}(s)}{\operatorname{N}(s)}$, heisst das Nennerpolynom $\operatorname{N}(s)$ auch charakteristisches Polynom. Durch Nullsetzen des charakteristischen Polynoms $\operatorname{N}(s)$ ergibt sich die charakteristische Gleichung des Systems $\operatorname{N}(s) = 0$.  Die Loesungen der charakteristischen Gleichung $s_{0_i}$ werden als Pole des Systems bezeichnet.
