% [p. 9]
Das Zustandsraummodell eines linearen zeitinvarianten (LTI) Systems ist gegeben durch die Zustandsgleichung	
$$\dot{x}(t) = A \operatorname{x}(t) + B \operatorname{u}(t)$$
und die Ausgabegleichung
$$\operatorname{y}(t) = C \operatorname{x}(t) + D \operatorname{u}(t)$$
Dabei ist $A \in \mathbb{R}^{n \times n}$ die Systemmatrix, $x \in \mathbb{R}^{n}$ der Systemzustandsvektor, $u \in \mathbb{R}^{m}$ der Eingangs- und $y \in \mathbb{R}^{q}$ der Ausgangsvektor. 
$B \in \mathbb{R}^{n \times m}$ heisst Eingangsmatrix, $C \in \mathbb{R}^{q \times n}$ Ausgangsmatrix und $D \in \mathbb{R}^{q \times m}$ Durchgangsmatrix.
