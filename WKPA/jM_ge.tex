Sei $\operatorname{f} : \mathcal{U} \subset \mathbb{R}^n \to \mathbb{R}^m$ eine partiell differenzierbare Funktion mit den Komponentenfunktionen $\operatorname{f}_i$, so dass $\operatorname{f} = [\operatorname{f}_1, \dots, \operatorname{f}_m]^T$ und bezeichne $x = \left[ x_1, x_2, \dots, x_n \right]^T$ die Koordinaten im Urbildraum $\mathbb{R}^n$. Dann ist die Jakobi Matrix $J_{\operatorname{f}}(a)$ im Punkt $a \in \mathcal{U}$ gegeben als
$$J_{\operatorname{f}}(a) :=  \begin{bmatrix}
	\frac{\partial \operatorname{f}_1}{\partial x_1}(a) & \frac{\partial \operatorname{f}_1}{\partial x_2}(a) & \ldots & \frac{\partial \operatorname{f}_1}{\partial x_n}(a) \\
	\vdots & \vdots & \ddots & \vdots \\
	\frac{\partial \operatorname{f}_m}{\partial x_1}(a) & \frac{\partial \operatorname{f}_m}{\partial x_2}(a) & \ldots & \frac{\partial \operatorname{f}_m}{\partial x_n} (a)
\end{bmatrix}$$
